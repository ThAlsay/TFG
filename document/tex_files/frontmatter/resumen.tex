\vskip 2cm

{\noindent\bfseries\Huge Resumen}

\vskip 1cm

Con el paso de los años el mundo docente se ha dado cuenta que el estudio de diversas materias cala mejor en el estudiante si se presenta de una manera amena e interactiva. En un mundo constantemente cambiante y cada vez más rápido, esto se ha visto reflejado en el aumento de la gamificación dentro del aula como una forma de combatir la constante perdida de atención y como una forma de que los conceptos se entiendan mejor y de forma clara.
\\

Por otro lado el cómputo distribuido cada vez toma más fuerza en esta nueva era en la que nos adentramos, dominada por los grandes modelos de inteligencia artificial y cada vez más conectada a la web. En este nuevo paradigma web impera la disponibilidad inmediata de los recursos, dando lugar al aumento inequívoco de los sistemas distribuidos.
\\

Con el fin de utilizar el juego como forma de aprendizaje de los sistemas distribuidos nace este proyecto, una librería en Elixir para la creación de un juego con varios escenarios para el aprendizaje de los alumnos.
\\

\textbf{Palabras clave:} Sistema Distribuido, Elixir, juego, aprendizaje.
